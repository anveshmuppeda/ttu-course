\documentclass[12pt, letterpaper]{article}

\usepackage{amssymb}
\usepackage{graphicx}
\usepackage{pictex}
\def \color#1]#2{}
\usepackage{palatino, url, multicol} % for multiple columns

\setlength{\topmargin}{-1.75cm} \setlength{\textheight}{22.5cm}
\setlength{\oddsidemargin}{0.25cm}
\setlength{\evensidemargin}{0.25cm} \setlength{\textwidth}{16.2cm}

\begin{document}

\newcommand{\ee}[1] {
  \begin{enumerate}
    #1
  \end{enumerate}
}
\newcommand{\ie}[1] {
  \begin{itemize}
    #1
  \end{itemize}
}

\newcommand{\hide}[1]{}
\newcommand{\set}[1]{\{#1\}}
\newcommand{\pg}[1]{{\tt #1}}
\newtheorem{definition}{Definition}

\newcommand{\emptyclause}{\Box}
\newcommand{\duedate}{
  05:00pm Mon, Oct 10}


%% \noindent test \hfill test

\title{{\bf Homework 3}. Tableau Proof. Resolution.}
\date{}
\maketitle

\input{hwHeader.tex}

\noindent Submit your solution in PDF file  and Latex source  file to blackboard by {\bf \duedate}.

\begin{enumerate}
% \item (double check ?? ) Write a complete final proof for Theorem 3.3. 
% (You have to construct the proof using backward approach on working paper. No need to submit your working backward procedure.) 
% Note that part of the proof is done by induction. Inside the proof by induction, we further use proof by cases.  
% \item For the definition of {\em tableaux from premises},  write the following information
\item (15)  For Definition 6.2,  write the following information in the order they occur in the definition
\begin{itemize}
 	\item For each concept defined by this definition, write its name and parameters (if there is any), 
 	\item For each concept used in this definition, write its name and  arguments   (if there is any), and
 	\item  Write meta variables in the definition.
 	
\end{itemize}
 
\item (15) For Definition 8.4,  write the following information in the order they occur in the definition
\begin{itemize}
 	\item For each concept defined by this definition, write its name and parameters (if there is any), 
 	\item For each concept used in this definition, write its name and  arguments   (if there is any), and
 	\item Write meta variables in the definition.
 	
\end{itemize}
 

%\item State and prove the soundness result of tableau proof of a proposition from a set of propositions. You may use the lemma in the book to prove this result. You need a complete proof in the form given in the proof of soundness of tableau proof at the end of L07. 

\item (15)  i) Find the definition of {\em assignment} from Chapter 8. Write the definition below. 

ii) Write the definition of another concept, whose name contains ``assignment", that was defined before (see L04). 

iii) Is it precise for us to understand {\em truth assignment} as the combination of the English meaning of truth and the definition of {\em assignment} in i)? Why? 


\item (15) i) Write the result of applying the definition of {\em satisfiable} (see Section 2.3 of L04). 
% Please follow the definition of {\em formula, satisfiable} and {\em assignment} of Chapter 8.
$$\set{\set{\neg A}, \set{A, \neg B}, \set{B}}.$$

ii) According to the result above, is the formula satisfiable? If yes, give an assignment
satisfying it. 

\item (20) Find a resolution tree refutation of 
% , i.e., resolution tree proof of $\emptyclause$ from,
the following formula:
$$\set{\set{A, \neg B, C}, \set{B, C}, \set{\neg A, C}, \set{B, \neg
C}, \set{\neg B}}.$$

\hide{
\item Given a formula $S = \set{ \set{A, B}, \set{C, B}, \set{D, \neg B}, \set{E, \neg B}, \set{F, C}}$.
What's the result of eliminating propositional letter $B$?
}

\item (20)  Prove that if the formula $S = \set{C_1, C_2}$ is satisfiable
and $C$ is the resolvent of $C_1$ and $C_2$, then $C$ is
satisfiable. Use our proof methodologies and format. Do not
copy the proof in the book.

%n\item Prove Lemma 5.4. Note do not copy the one in the book. Please construct the proof following the general proof techniques we used in the class.

\end{enumerate}

\end{document}
