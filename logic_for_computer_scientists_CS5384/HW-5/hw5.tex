

\documentclass[12pt, letterpaper]{article}

\setlength{\topmargin}{-1.75cm} \setlength{\textheight}{22.5cm}
\setlength{\oddsidemargin}{0.25cm}
\setlength{\evensidemargin}{0.25cm} \setlength{\textwidth}{16.2cm}

\usepackage{amssymb}
\usepackage{graphicx}

\usepackage{palatino, url, multicol} % for multiple columns

\usepackage{pictex}
%% in the .pictex output of xfig, there is command \colo
%% however the old version of pictex may not define this
%% so we define color here as empty
\def \color#1]#2{}

\newcommand{\hwnumber} {
    5
}

\newcommand{\hwtitle} {
   Predicate Logic: Syntax and Semantics
}

\newcommand{\duedate}{
    11:59pm Sun Nov 20}
\date{}
\begin{document}

\newcommand{\hide}[1]{}
\newcommand{\otherquestions}[1]{}
\newcommand{\set}[1]{\{#1\}}
\newcommand{\pg}[1]{{\tt #1}}
\newtheorem{definition}{Definition}
\newcommand{\emptyclause}{\Box}


%% \noindent test \hfill test


\title{{\bf Homework \hwnumber.} \hwtitle}

\date{}
\maketitle

\input{hwHeader.tex}

\noindent 
Submit your solution in PDF file (and Latex source  file if you use Latex) to blackboard by 
{\bf \duedate}. 

\begin{enumerate}
\item Consider the following sentence.

{\em Every number greater than or equal to 4 can be written
        as the sum of two prime numbers.}

  \begin{enumerate}
    \item Write {\em a language} (as defined in Def 2.1) such that
    some formulas of the language can be used to represent the sentence above.

    \item Write a formula of your language that should reflect the
    meaning of the sentence above.

    \item In terms of your language, write three example terms, three example atomic
    formulas, and three example formulas.
  \end{enumerate}

\item Write a formula to represent the following information. Your formula should be as close as possible
to the intended meaning of these sentences.

\begin{enumerate}
\item {\em There is a mother to all children.}

% from Unit Lo Logic http://cr.yp.to/2005-261/bender1/Lo.pdf
\item ALL ITEMS NOT AVAILABLE AT ALL STORES.

Note, the sentence above is a disclaimer in the weekly flyer of
specials of a grocery store chain.
\end{enumerate}

\item Given the language defined in Def 2.1, which of the following
are formulas defined by Def 2.5?

  \begin{enumerate}
    \item $f(x,c)$
    \item $R(c, f(d,z))$
    \item $\forall x (P(x))$
    \item $((\exists x)(((\forall y)P(z)) \rightarrow R(x,y)))$
  \end{enumerate}
\item Given

$$((\exists x)(((\forall y)P(z)) \rightarrow R(x,y))),$$
  \begin{enumerate}
    \item List all its subformulas.
    \item Draw its formation tree.
  \end{enumerate}

\item Which of the following terms are substitutable for $x$ in the
corresponding formulas?
  \begin{enumerate}
    \item $f(z,y)$ in $((\exists y)(P(y) \land R(x,z))).$
    \item $g(f(z,y),a)$ in $((\exists x)(P(x) \land R(x,y))).$
  \end{enumerate}

\item Connect predicate calculus to the study of this course. To focus on the substance, we need to extend (informally) predicate calculus (syntax) as follows  
  \begin{itemize} 
     \item You can use sets or proposition as a parameter of a predicate. 
     \item You can use $=$ as a predicate symbol in the normal way. For example, that two terms $t_1$ and $t_2$ are equal can be represented by $t_1 = t_2$. 
     \item You can use a variable to refer to a function or a set or a proposition. (e.g., $((\exists \mathcal{V}) \mathcal{V}(x) = T)$ where $T$ is constant.)
  \end{itemize}	
  The form of formula will be expanded accordingly to the extension above. 
  Consider the definition of {\em consequence} (Def 3.7 of Part I).  
  \begin{enumerate}
  	\item Represent it as a formula. You need to introduce all predicates you need in the formula,  in the way we did in L11.1 for the subset example. Intuitively, a concept name and its parameters/arguments you identify in the definition is a good candidate of a predicate. You also need to introduce any constant or function symbol you need.  Note the main ``if" in a definition should be understood as ``if and only if." 
  	Note statement ``$\forall x \in A, x \in B$" in the subset definition. We translate it to ``$((\forall x)(x \in A \rightarrow x \in B))$."
  	\item Let your formula be $\alpha$. What can you logically derive (some intuition/experience from your earlier study is needed here) from the formula ($\alpha \land$ ``B is a consequence of $\{A, A \rightarrow B\}$")? Note you should translate the English in the formula using predicate(s) you introduced. Your answer to this question has to be in the form of a formula. You need to do a variable substitution ($\Sigma$ in English definition would be replaced by $\{A, A \rightarrow B\}$, and $\sigma$ by $B$). 
  \end{enumerate} 
\item 
1) Let $A = \{1, 2\}$. a) List all functions from $A$ to $A$ in the form of sets of pair. For example, one function is $\{(1,1), (2,2)\}$. If we let the function be named $g$. Then in the example function, $g(1)$ is 1 and $g(2)$ is 2.  b) List all unary relations on $A$. Your relations must be represented as sets.

2) Show that $\forall x (p(x) \rightarrow q(f(x))) \wedge \forall x
p(x) \wedge \exists x \neg q(x)$ is satisfiable. The format of your structure should follow the one in L12.1. In your structure, you must use the domain 
$A = \{1, 2\}$. You must represent the function assigned to $f$ in the set of pairs. You must represent the relations assigned to $p$ and $q$ in the set forms.  Note you have to figure out the constants in the language on which the formula is defined. 

\item Prove that $\mathcal{A} \models \neg \exists x
\varphi(x)$ if and only if $\mathcal{A} \models \forall x \neg \varphi(x)$.  
You have to follow the proof format we used earlier. Working backward again is a good idea. You have to be able to apply the definitions. 

\otherquestions{
  \item Consider the following definition.

  \begin{definition}
    A set $A$ is a subset of  a set $B$ if for all $x$ of $A$, $x
    \in B$.
  \end{definition}
    \begin{enumerate}
      \item Write {\em a language} such that some formulas of the
  language can be used to describe the definition above.

      \item Write a formula to represent the definition above.
      \item Assume you expand your language with function symbol $\times$.
      What's the result of the substitution $A/A\timesB$ of your
      formula in (b).
    \end{enumerate}
  \item Using Theorem 2.14 to prove that the number of right and left parentheses are equal in every formula.
}

\end{enumerate}

\end{document}
